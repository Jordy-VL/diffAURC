\begin{center}{\noindent\Large\bf Details on AURC}\end{center}

\section{Standard Definition of Selective Classification} %g is independent of model parameters w

Given a data set $\left\{\left(x_i, y_i\right)\right\}_{i=1}^N$ of size $N$ with $\left(x_i, y_i\right)$ independent samples from $\mathcal{X} \times \mathcal{Y}$, and given a pair of functions $(f, g)$, where $g: \mathcal{X} \rightarrow \mathbb{R}$ is a confidence scoring function (CSF) and $f(\cdot): \mathcal{X} \rightarrow \mathcal{Y}$ is the classifier, the classification output after failure detection is defined as:
$$
    (f, g)(x):=\left\{\begin{array}{ll}
        f\left(x\right),  & \text { if } g(x) \geq \tau \\
        \text { filter, } & \text { otherwise }
    \end{array}\right.
$$
Filtering ("detection") is triggered when $g(x)$ falls below a threshold $\tau \in \mathbb{R}$.
In order to perform \textit{meaningful} failure detection, a CSF $g(x)$ is required to output high confidence scores for correct predictions and low confidence scores for 'wrong' predictions based on the binary failure label
$$
    y_{\mathrm{w}}(x)=\mathbb{I}\left[y \neq \hat{y}\right],
$$
where for convenience, $f_{k}(x)$ denotes the $k$-th element of the output vector, thus $\hat{y}=\operatorname{argmax}_{y' \in \mathcal{Y}} f_{y'}(x)$ and $\mathbb{I}$ is the identity function (1 for true events and 0 for false events).


\begin{definition}[Confidence Scoring Function (CSF)]
    Any function whose continuous output \textit{aims to} separate a model's failures from correct predictions can be interpreted as a confidence scoring function. \cite{jaeger2023a}.
    \begin{equation}
        g(x) \in \mathbb{R} \text { s.t. } \left\{\begin{array}{ll}
            g(x) \geq \tau, & \text { if } y_{\mathrm{w}}(x)=0 \\
            g(x) < \tau,    & \text { if } y_{\mathrm{w}}(x)=1
        \end{array}\right.
    \end{equation}
\end{definition}

The goal of the CSF is to induce a partial order over points in $\mathcal{X}$, with optimally reflecting true loss monotonicity in the sense that for every two labeled instances $\left(x_1, y_1\right) \sim P(X, Y)$, and $\left(x_2, y_2\right) \sim P(X, Y)$,
$$
    g\left(x_1, \hat{y} \mid f\right) \leq g\left(x_2, \hat{y} \mid f\right) \Longleftrightarrow \operatorname{Pr}_P\left[\hat{y}_w\left(x_1\right) \neq y_1\right] \geq \operatorname{Pr}_P\left[\hat{y}_w\left(x_2\right) \neq y_2\right]
$$
%the confidence of 1 is lower than 2 if if the probability of failure of 1 is higher than 2; where failure is indicated by the true probability of the failure label $y_{\mathrm{w}}(x)=\mathbb{I}\left[y \neq \hat{y}\right]$.

\noindent Common CSFs used in practice \footnote{While it is preferable to have the output domain of g in $[0,1]$ for easier finding of selection thresholds, this is not a strict requirement.} are:
\begin{enumerate}
    \item Maximum softmax probability (MSP): $g(x) = \max_{y' \in \mathcal{Y}} f_{y'}(x)$
    \item Maximum logit (MaxLogit): $g(x) = \max_{y' \in \mathcal{Y}} z_{y'}(x)$, with \textit{logits} $\mathbf{z} \in \mathbb{R}^K$
    \item Entropy: $g(x) = -\sum_{y' \in \mathcal{Y}} f_{y'}(x) \log f_{y'}(x)$
    \item Distance-based measures (e.g., \cite{fisch2022calibrated})
          \begin{itemize}
              \item kNN distance. A 1D outlier score derived from the average distance of $x$ to its $k$ nearest neighbors in the training distribution
          \end{itemize}
          % \item Mutual information (MI): $g(x) = \sum_{y' \in \mathcal{Y}} f_{y'}(x) \log \frac{f_{y'}(x)}{p(y')}$, where $p(y')$ is the prior probability of class $y'$
          % \item Predictive entropy (PE): $g(x) = -\sum_{y' \in \mathcal{Y}} f_{y'}(x) \log \sum_{y'' \in \mathcal{Y}} f_{y''}(x)$
          % \item Predictive entropy reduction (PER): $g(x) = \sum_{y' \in \mathcal{Y}} f_{y'}(x) \log \frac{f_{y'}(x)}{\sum_{y'' \in \mathcal{Y}} f_{y''}(x)}$
          % \item Distance to decision boundary (D2B): $g(x) = \frac{1}{\norm{\mathbf{w}}_2} \left(\mathbf{w}^T \mathbf{x} + b\right)$, where $\mathbf{w}$ and $b$ are the weights and bias of the last layer of the NN
          %https://github.com/ajfisch/calibrated-selective-classification/blob/main/src/data/features.py 
\end{enumerate}

While it is preferable to have the output domain of g in $[0,1]$ for easier finding of selection thresholds, it is not a strict requirement.

\section*{Jordy's AURC}

A crucial component in making a multi-class classifier $f$ selective is the selector function $g$, which originally defined as a CSF, requires a threshold $\tau$ to be defined in order to perform filtering. The threshold $\tau$ is usually defined as a fixed value, or as a function of the confidence score distribution of the training set.
With the objective of measuring a classifier's selective performance, we need to establish risk and coverage measures that are dependent on the threshold $\tau$.

\begin{definition}[Coverage]
    The coverage of a selective classifier $f$ with selector $g$ and threshold $\tau$ is defined as the expectation of selecting from a distribution.
    \begin{equation}
        \mathcal{C}_{\tau}(f, g)=\mathbb{E}_{P_X}\left[g(x) \geq \tau\right]
    \end{equation}
    Put plainly: the percentage of samples above the threshold %n_samples_above_threshold / n_samples
    % probability mass of the non-rejected region in \mathcal{X} 
\end{definition}

\begin{definition}[Risk]
    The risk of a selective classifier $f$ with selector $g$ and threshold $\tau$ is defined as the expected loss of the classifier on the filtered distribution
    %$\mathcal{D}_{\text {in }}^{\text {test }, \times}$, i.e. the expected loss of the classifier on the set of misclassified in-distribution test samples.
    \begin{equation}
        \mathcal{R}_{\tau}(f, g)=\frac{\mathbb{E}_{P_{XY}}\left[\ell\left(y, f(x)\right) \mathbb{I}\left[g(x) \geq \tau\right]  \right]}{\mathcal{C}_{\tau}(f, g)} %
    \end{equation}
    Put plainly: the percentage of errors above the threshold %residuals / n_samples
\end{definition}

To estimate the area under the risk-coverage curve \AURC{}, a multi-threshold list $\theta_T$ with length $T$ is obtained as the unique values of the ascending ranking of all CSF values $g(X)$. The $\mathrm{AUC}$ is then approximated by the sum of the areas of the trapezoids defined by the points $(\mathcal{C}_{\tau}(f, g), \mathcal{R}_{\tau}(f, g))$ for all thresholds $\tau \in \theta_T$.

%Starting point is risk=1-accuracy, coverage=100\%; Ending point= coverage=0\%, risk=0\%.


%(measured on a given distribution, e.g., the validation set)

\section{End-to-End example of AURC}

%Despite its relevance for application, the often final step of failure detection, i.e. the definition of a decision threshold on the confidence score, is often neglected in research. In Appendix D we present an approach that does not require the calibration of scores; optimize coverage, relative to maximal risk

%\section{Detail on integration for AURC}


\section*{Why AURC?}

Using AURC as a primary evaluation metric has the following advantages \cite{ding2020revisiting}.
\begin{enumerate}
    \item when the underlying prediction models are the same, AURC is an effective quality metric to indicate the performance of selective prediction
    \item  when the prediction models are different, which happens a lot in the literature due to the emerging trend of uncertainty-aware training (or swapping out loss functions), using AURC instead of AUPR and AUROC prevents unfair and potential misleading comparison.
    \item AURC can be generalized to distinct tasks with task-specific evaluation metrics ($\ell(y,f(x))$) while AUPR and AUROC cannot.
    \item AURC is an alternative optimization objective to directly maximize the performance of selective prediction and helps to avoid weighing multiple objective terms in related work
    \item AURC directly shows the cost-performance trade-off in selective prediction which is not visible in the conventional ROC curve or PR curve.
\end{enumerate}

\newpage
