%\title{Motivating selective classification for document classification} %\project
%\maketitle
\begin{center}{\noindent\Large\bf Defining Selective Multi-Class Classification}\end{center}

\section{AURC definition}

Given a multi-class classifier $f: \mathcal{X} \rightarrow \Delta^{k} \times [0,1]$ for a $k$ class problem with a reject option, where $\Delta$ is a probability simplex. %assumes additional logit; theta form part

\noindent Origin: \url{https://arxiv.org/pdf/1805.08206.pdf}

\begin{align}
    \operatorname{AURC}(f) := \mathbb{E}_{x \sim p_x} \left[  \frac{\mathbb{E}_{(\tilde{x},\tilde{y})\sim p_{xy}}[\ell([f(\tilde{x})]_{1:k},\tilde{y}) \mathbb{I}[f(\tilde{x})_{k+1} > f(x)_{k+1}]]}{\mathbb{E}_{\tilde{x}\sim p_{x}}[ \mathbb{I}[f(\tilde{x})_{k+1} > f(x)_{k+1}]]} \right] \\
    \approx \frac{1}{n} \sum_{h=1}^n \frac{\frac{1}{n-1} \sum_{i\neq h} \ell([f(x_i)]_{1:k},y_i) \mathbb{I}[f(x_i)_{k+1} > f(x_h)_{k+1}]}{\frac{1}{n-1} \sum_{i\neq h} \mathbb{I}[f(x_i)_{k+1} > f(x_h)_{k+1}]}
\end{align}

\section{Example \AURC}


\newpage

\begin{center}{\noindent\Large\bf Motivating Selective Document Classification}\end{center}

\begin{uuredbox}
This version (18/09) has been stated in terms of the context and motivation, not yet as properly formulated definitions; bear with me on this :) 
\end{uuredbox}

\section{Context}

Document classification (\DC) is a quintessential task in intelligent document processing workflows. In practice, it is characterized as a multi-class classification problem with typically high cardinality (100-500 mutually exclusive document types) \cite{vanlandeghem2023document}. Production-level \DC{} models operate in a high-accuracy regime (>80\%) and are expected to be both \textit{classification-calibrated} \cite[Definition 1]{ni2019calibration} (top-1 calibration \cite{naeini2015obtaining,guo2017calibration,vaicenavicius2019evaluating}) and \textit{rejection-calibrated} \cite[Definition 2]{ni2019calibration}. As the predicted document type dictates further processing of key information extraction (KIE), typically specified according to static formats (e.g., a receipt for expenditure specifies different information than a car collision form), this primary step in the pipeline is crucial. 

%not SOTA though 

\section{Motivation}

There is a lot of interest in reducing overall risk while maintaining the highest possible target coverage in large multi-class classification problems. Naturally, rejecting low-confidence, uncertain samples for human intervention is preferred to trusting a classifier. 
This is the target of interest, whereas we train our models with a differentiable objective (commonly cross-entropy loss $\ell_{\mathrm{CE}}$) that might not be aligned. \cite{geifman2017selective,geifman2018bias} point out that there is no consensus on the appropriate performance measure for scoring functions.
In practice, the classes can be imbalanced and might have a higher associated misclassification cost, although users are not likely to quantify class-specific costs for a label set of high cardinality. %there is also a link with fairness, as some document types might be more privy to processing 
An additional practical problem exists in finding an appropriate set of class-specific thresholds that overall minimizes the target risk and maximizes a high (>70\%) target coverage. 

In the \DC{} task setting, it is more sensible to target low-risk levels, instead of targeting midpoint coverage rates. Therefore, I propose to research a surrogate loss function with additional control over the risk, incentivizing the model to prioritize the true rejection rate. 

%TODO: formulate correctly as definitions 

\section{Background}

\subsection{Calibration}

\todo[inline]{Further establish link/difference to calibration}


\noindent \begin{uuyellowbox}[title=Top-1 calibration does not imply good failure prediction \cite{zhu2022rethinking}]

    Consider two probabilistic classifiers, each providing max softmax probability (MSP) as the confidence scoring function (CSF).\newline

    Classifier A: 90\% accuracy - varying MSP ranging from 0.8-1
    
    Classifier B: 92\% accuracy - MSP always 0.92\newline

    Which classifier would you choose in practice (keeping in mind the goals of selective classification)?
    
\end{uuyellowbox}



\subsection{Selective classification}

It has been referred to as confidence ranking in the context of \textit{iid} misprediction classification or failure/error prediction \cite{jaeger2023a}. For a good overview, check the survey \cite{zhang2023survey}. 

\noindent Popular \textbf{evaluation metrics} include:
\begin{enumerate}
    \item \AUROC{} (Area Under the ROC Curve)
    \item \AURC{} (Area Under the Risk Coverage Curve) \citep{geifman2017selective,jaeger2023a} 
    \item \EAURC{} (discounting accuracy and normalization) \citep{geifman2018bias}
\end{enumerate}


\noindent Similar to calibration research, there are \textbf{I. train-time} methods involving surrogate losses, architecture \& optimization extensions, and \textbf{II. training-free} methods.
\todo[inline]{Add related work, \href{https://openreview.net/pdf?id=K1DdnjL6p7}{recent reference}}

\noindent Important \textbf{theoretical result}:

Impossibility of "separation-based" calibrated surrogates for multi-class classification \cite[Theorem 4]{ni2019calibration}, making it a must-read. If I understand it well enough, it posits that $\ell_{\mathrm{CE}}$ is the best available choice. 

Another important related work \href{https://arxiv.org/pdf/2208.12084.pdf}{Calibrated Selective Classification} \cite{fisch2022calibrated} proposes a trainable objective for selective calibration based on MMCE \cite{kumar2018trainable} \footnote{There is \href{https://github.com/ajfisch/calibrated-selective-classification}{code}}.


\subsection{Knowledge distillation}

Teacher-student training involves soft labels, which AFAIK has not been investigated together with selective classification. Matthew suggested swapping out $\ell_{\mathrm{CE}}$ for a proposed surrogate (\Cref{sec:method}) while fine-tuning the teacher and/or student network. 
Recent work \cite{galil2023what} showed that models trained with knowledge distillation demonstrated improved \AURC{}.


\section{Method}\label{sec:method}

\subsubsection{Prior methods}

\cite{geifman2017selective,geifman2018bias} provides the seminal paper with a computationally heavy method (ad-hoc loss and architecture extension) to control coverage and risk at training time.
%for training a neural network with three heads: a classification head, and auxiliary head, and one head to estimate the confidence of the classification decision.

Regarding I. it is hard to establish a clear winning method (in terms of \AURC) over MSP, the literature has shown that differences are often not reproducible\footnote{\href{https://github.com/LayneH/SAT-selective-cls/issues/3\#issuecomment-1640151933}{Discussion} ongoing.} or not statistically significant \cite{jaeger2023a}.
The current (II.) state-of-the-art \cite{lin2022scrib} treats it as a set-based prediction with a post-hoc multi-thresholding algorithm that reduces overall risk. They devised an algorithm based on coordinate ascent (check 1 variable, keep others fixed) with some efficiency optimization at the lower level of class-specific thresholding. 



\todo[inline]{Describe the optimal solution with time/space complexity (Big-Oh) and naive implementation}

\todo[inline]{Describe a baseline solution}

% \subsection{Task description}

% \subsection{Method contributions}

% \begin{itemize}
% \item 
% \end{itemize}


\section{Experiments}

I have created a benchmark (compatible with HuggingFace) for \KD{} on \DC{}.


\subsection{Selective Training Methods}

For teacher fine-tuning, the following (I.) methods have been implemented: 

\begin{enumerate}
    \item $\ell_{\mathrm{CE}}$ baseline 
    \item Confidence Ranking Loss \cite{moon2020confidence}
    \item LogitNorm Loss \cite{wei2022mitigating}
    \item Deep Gamblers Loss \cite{liu2019deep}
    \item Symmetric Cross-Entropy \cite{wang2019symmetric}
    \item Entropy Regularization \cite{feng2022stop} 
    \item Self-adaptive Training \cite{huang2020self}
    \item Intermediate Checkpointing \cite{wang2022efficient}
\end{enumerate}

Other interesting (II.) training-free methods to be investigated (\#TODO): 
\begin{itemize}
    \item DOCTOR \cite{granese2021doctor}
    \item Gini \cite{gomes2023a}
    \item SCRIBR \cite{lin2022scrib}
\end{itemize}

\subsection{Distillation Methods}

Current \KD{} methods are supported: 
\begin{itemize}
\item  Response/logits-based
\begin{enumerate}
    \item Cross-entropy with KD hyperparameters ($\alpha$ and $\tau$) \cite{hinton2015distilling}
    \item Mean Squared Error \cite{kim2021comparing}
    \item NKD \cite{yang2023knowledge}
    %\item Need something different for FasterRCNN
    %\item (Hyperparameter grid search run for all)
\end{enumerate}
\item  Features-based
\begin{enumerate}
    \item Distillation on reconstructed feature maps (FitNet) - middle layer strategy \cite{romero2014fitnets}
    \item SimKD with re-used teacher head \cite{SimKD} %Patch to patch distillation framework 
    \item SimKD with [CLS] MLP projector variation
    %\item Structural knowledge distillation
\end{enumerate}
\end{itemize}



\subsection{Datasets}

\begin{table}[ht]
\centering
\caption{Dataset usage for DIC and DLA tasks. 
Symbols: P = pre-training, DP = document pre-training, T = teacher training, S = student training, * = subsampling , E = teacher/student evaluation.}
\begin{tabular}{|l|l|c|c|c|}
\hline
\textbf{Dataset} & \textbf{Task} & \textbf{Usage} & \textbf{Size} & \textbf{\# Cls} \\ \hline
ImageNet \cite{deng2009imagenet} & DIC & P & 1.28M & 1000 \\ \hline
IIT-CDIP \cite{lewis2006building} & DIC & DP, T, S & 11M & / \\ \hline
\tobacco{} \cite{kumar2013unsupervised} & DIC & T, S, E & 3482 & 10 \\ \hline
\rvl{} \cite{harley2015evaluation} & DIC & DP, T, E, * & 400K & 16 \\ \hline
RVL-CDIP-N \cite{larson2022evaluating} & DIC & E & 10K & 16 \\ \hline
\hline
\end{tabular}
\label{tab:DKD-datasets}
\end{table}

For \DC{}, we benchmark results on both \tobacco{} (original train-val-test splits 800-200-2482) and \rvl.  
We also created a subsampled student training set, \rvlone, by randomly selecting 1K training images per class from \rvl. The originally large training size hinders experimentation (longer iteration cycles).
By evaluating the full \rvl{} test set, we provide a fair evaluation of the usefulness of KD methods, while avoiding the cumbersomeness of student fine-tuning on a large dataset. Additionally, we evaluate RVL-CDIP-N, which is a test set created for testing robustness to covariate shifts.

\todo[inline]{
METHODOLOGY PROBLEM: no open-source data to test document classification with high label cardinality.
An internal dataset is available.
}



\subsubsection{Preliminary results varying KD strategies - CE teacher}

\begin{table*}
\centering
\caption{Preliminary results of different KD strategies benchmarked for Vit-B applied on the \rvl{} datasets. }
\npdecimalsign{.}
\nprounddigits{3}
\begin{tabular}{|c|c|c|c|n{1}{3}n{1}{3}n{1}{3}|} %{|c|c|c|c||ccc|}  
\hline Dataset & Teacher backbone & Student backbone & Method & \text{ACC} & \text{ECE} & \text{AURC}\\  \hline %$\triangle \mathrm{Acc}$
\rvl & ViT-B & -- & Baseline & 0.891697292432311 &	0.047907298836262	& 0.017503164554713 \\
& -- & ViT-S & Baseline &0.853371334283357 & 0.057859288371025 & 0.02963421682364 \\ %0.90844771119278 & 0.01186920564432 & 0.012508771541678 \\
 & -- & ViT-T & Baseline &0.822045551138778 & 0.042795402533132 & 0.040342312570416 \\ \hline %0.900922523063077 & 0.016201313786121 & 0.014282554880225 \\
\hline \rvlone  & ViT-B & \textbf{ViT-S} & Vanilla [$\tau=2.5, \alpha=0.5$] &0.85427135678392 & 0.048633268034969 & {\npboldmath}0.028376044015258 \\
\hline \rvlone  & ViT-B &  & NKD [$\tau=1, \gamma=1.5$] &0.840471011775294 & 0.073632049497662 & 0.035729143933197 \\
\hline \rvlone  & ViT-B &  & MSE &0.854996374909373 & 0.050951811632588 & {\npboldmath}0.028086462460537 \\
\hline \rvlone  & ViT-B &  & SimKD [CLS+MLP]  & {\npboldmath}0.85947148678717 &	{\npboldmath}0.028154179258953 &	0.28744054887619
 \\
\hline \rvlone  & ViT-B &  & SimKD [CNN] & 0.846796169904247	&0.061565487248227	& 0.141136871882953
 \\
\hline \rvlone  & ViT-B &  & FitNet [middle] &0.842646066151654 & 0.140569373835247 & 0.047789833130271 \\
\hline \rvlone  & ViT-B (ImageNet) & ViT-T & FitNet [middle] & 
0.825895647391185	&0.056582912881468	& 0.155618612805129
 \\
\hline\hline \rvlone  & ViT-B & \textbf{ViT-T} & Vanilla [$\tau=2.5, \alpha=0.5$] &  0.824745618640466 & 0.057973819137046 & {\npboldmath}0.03840438493781 \\
\hline \rvlone  & ViT-B &  & NKD [$\tau=1, \gamma=1.5$]& 0.815070376759419 & 0.093624998224256 & 0.045976107157128 \\
\hline \rvlone  & ViT-B &  & MSE & 0.82329558238956 & 0.065716418501806 & 0.039916989542324 \\
\hline \rvlone  & ViT-B &  & SimKD [CLS+MLP] & {\npboldmath}0.829745743643591 & 0.163239800045468 & 0.094989035012674 \\
\hline \rvlone  & ViT-B &  & SimKD [CNN] & 0.829495737393435 & 0.149636502763522 & 0.055740155161562 \\
\hline \rvlone  & ViT-B &  & FitNet [middle] & 0.812345308632716 & 0.153136880692877 & 0.050564425673713 \\
\hline \rvlone  & ViT-B (ImageNet) & ViT-T & FitNet [middle] & 0.805370134253356	&{\npboldmath}0.05986580106555&	0.167935470434519
 \\
\hline
\end{tabular}
\end{table*}


% Considering standard Neural Networks (NNs), the last layer outputs a vector of real-valued \textit{logits} $\mathbf{z} \in \mathbb{R}^K$, which in turn are normalized using a sigmoid/softmax activation function.

% $\displaystyle \sigma(z) = \frac{1}{1 + \exp^{-z}}$
% $\displaystyle \text{softmax}(\mathbf{z}) = \frac{{\exp(z)}}{{\sum_{k=1}^{K} \exp(z_k)}}$

% For convenience, $f_{k}(x)$ denotes the $k$-th element of the output vector.

% $\hat{y} = \argmax_{y'\in{\mathcal{Y}}}f_{y'}(X)$ is the top-1 class prediction

% $\hat{p}= \max_{y'\in{\mathcal{Y}}}f_{y '}(X)$ is the associated posterior probability
% Some interesting distributions are defined:

% \begin{itemize}
%     \item $\mathcal{D}_{\text {in }}$ denotes the distribution over $\mathcal{X}$ of in-distribution (ID) data
%     \item 	 $\mathcal{D}_{\text {in }}^{\text {test }, \checkmark}$ and $\mathcal{D}_{\text {in }}^{\text {test }, \times}$ represent the distribution of correct and misclassified ID test samples
% \end{itemize}

% Now to combine these two measures into a single metric, we need to define a trade-off between risk and coverage. This is done by defining a \textit{cost} function $c(\cdot)$, which is a function of the risk and coverage. The cost function is defined as the weighted sum of the risk and coverage, where the weights are defined by the user. The cost function is defined as:
% \begin{equation}
%     c(\mathcal{R}_{\tau}(f, g), \mathcal{C}_{\tau}(f, g))=\alpha \mathcal{R}_{\tau}(f, g)+(1-\alpha) \mathcal{C}_{\tau}(f, g)
% where $\alpha \in[0,1]$ is a user-defined parameter that controls the trade-off between risk and coverage. The cost function is a function of the threshold $\tau$ and the selector $g$.
